\section{插值与拟合习题}
\begin{tcolorbox}[breakable,enhanced,arc=0mm,outer arc=0mm,
		boxrule=0pt,toprule=1pt,leftrule=0pt,bottomrule=1pt, rightrule=0pt,left=0.2cm,right=0.2cm,
		titlerule=0.5em,toptitle=0.1cm,bottomtitle=-0.1cm,top=0.2cm,
		colframe=white!10!biru,colback=white!90!biru,coltitle=white,
            coltext=black,title =2024-03-10, title style={white!10!biru}, before skip=8pt, after skip=8pt,before upper=\hspace{2em},
		fonttitle=\bfseries,fontupper=\normalsize]
  
1. 已知等距插值节点,
$$
x_{0}<x_{1}<x_{2}<x_{3}, \quad x_{i+1}-x_{i}=h \quad({i}=0,1,2)
$$
且 $ f(x) $ 在 $ \left[x_{0}, x_{3}\right] $ 上有四阶连续导数, 证明 $ f(x) $ 的 Lagrange 插值多项式余项的误差界为

(1) 二次插值的误差界
$$
R_{2}=\max _{x_{0} \leq x \leq x_{2}}\left|f(x)-L_{2}(x)\right| \leq \frac{\sqrt{3}}{27} h^{3} \max _{x_{0} \leq x \leq x_{2}}\left|f^{\prime \prime \prime}(x)\right|
$$
(2) 三次插值的误差界
$$
R_{3}=\max _{x_{0} \leq x \leq x_{3}}\left|f(x)-L_{3}(x)\right| \leq \frac{1}{24} h^{4} \max _{x_{0} \leq x \leq x_{3}}\left|f^{(4)}(x)\right|
$$
 \tcblower

(1) 二次插值的误差界:

首先,我们知道二次 Lagrange 插值多项式为
$$
L_{2}(x)=f\left(x_{0}\right) \frac{(x-x_{1})(x-x_{2})}{(x_{0}-x_{1})(x_{0}-x_{2})}+f\left(x_{1}\right) \frac{(x-x_{0})(x-x_{2})}{(x_{1}-x_{0})(x_{1}-x_{2})}+f\left(x_{2}\right) \frac{(x-x_{0})(x-x_{1})}{(x_{2}-x_{0})(x_{2}-x_{1})}
$$

定义余项为 $ R_2(x)=f(x)-L_{2}(x) $,则存在 $ \xi \in(x_{0}, x_{2}) $,使得
$$
R_2(x)=\frac{f^{\prime \prime \prime}(\xi)}{3 !}(x-x_{0})(x-x_{1})(x-x_{2})
$$
令 $ x=x_{0}+t h $ ,因此,我们有


$$
\begin{aligned}
\max _{x_{0} \leqslant x \leqslant x_{2}} |R_2(x)| &=\max _{x_{0} \leqslant x \leqslant x_{2}}\left|\frac{f^{\prime \prime \prime}(x)}{6}(x-x_{0})(x-x_{1})(x-x_{2})\right| \\
&\leqslant \frac{1}{6} \max _{x_{0} \leqslant x \leqslant x_{2}}\left|f^{\prime \prime \prime}(x)\right| \cdot \max _{x_{0} \leqslant x \leqslant x_{2}}\left|\left(x-x_{0}\right)\left(x-x_{1}\right)\left(x-x_{2}\right)\right| \\
&=\frac{1}{6} h^{3} \max _{0 \leqslant t \leqslant 2}|t(t-1)(t-2)| \cdot \max _{x_{0} \leqslant x \leqslant x_{2}}\left|f^{\prime \prime \prime}(x)\right| \\
\end{aligned}
$$

令$g(t)=t(t-1)(t-2)=t^3-3t^2+2t, \quad 0 \leqslant t \leqslant 2$.
令$ g^{\prime}(t)=0 $, 即 $ 3 t^{2}-6 t+2=0 $, 解得
$$
t_{1}=1-\frac{1}{\sqrt{3}}, \quad t_{2}=1+\frac{1}{\sqrt{3}}
$$
由 $ g(0)=0, g\left(t_{1}\right)=\frac{2 \sqrt{3}}{9}, g\left(t_{2}\right)=-\frac{2 \sqrt{3}}{9}, g(2)=0 $知$\max\limits _{0 \leqslant t \leqslant 2} |g(t)|=\dfrac{2 \sqrt{3}}{9}$.于是
$$
R_{2}=\max _{x_{0} \leqslant x \leqslant x_{2}}|R_2(x)|=\max _{x_{0} \leqslant x \leqslant x_{2}}\left|f(x)-L_{2}(x)\right| \leqslant\frac{1}{6} h^{3}\cdot\frac{2 \sqrt{3}}{9} \max _{x_{0} \leqslant x \leqslant x_{2}}\left|f^{\prime \prime \prime}(x)\right|=\frac{\sqrt{3}}{27} h^{3}  \max _{x_{0} \leqslant x \leqslant x_{2}}\left|f^{\prime \prime \prime}(x)\right|
$$

(2) 三次插值的误差界:

三次 Lagrange 插值多项式为
$$
L_{3}(x)=f\left(x_{0}\right) \frac{(x-x_{1})(x-x_{2})(x-x_{3})}{(x_{0}-x_{1})(x_{0}-x_{2})(x_{0}-x_{3})}+\cdots
$$

定义余项为 $ R_3(x)=f(x)-L_{3}(x) $,同理可以得到
$$
R_{3}= \max _{x_{0} \leqslant x \leqslant x_{3}}|R_3(x)|=\max _{x_{0} \leqslant x \leqslant x_{3}}\left|f(x)-L_{3}(x)\right| =\max _{x_{0} \leqslant x \leqslant x_{3}}\left|\frac{f^{(4)}(x)}{4 !}\left(x-x_{0}\right)\left(x-x_{1}\right)\left(x-x_{2}\right)\left(x-x_{3}\right)\right|
$$


令 $ x=x_{0}+t h $ ,则

$$
\begin{aligned}
R_3 & =\max _{x_{0} \leqslant x \leqslant x_{3}}\left|\frac{f^{(4)}(x)}{4 !}\left(x-x_{0}\right)\left(x-x_{1}\right)\left(x-x_{2}\right)\left(x-x_{3}\right)\right| \\
& \leqslant \frac{1}{24} \max _{x_{0} \leqslant x \leqslant x_{3}}\left|f^{(4)}(x)\right| \cdot \max _{x_{0} \leqslant x \leqslant x_{3}}\left|\left(x-x_{0}\right)\left(x-x_{1}\right)\left(x-x_{2}\right)\left(x-x_{3}\right)\right| \\
& =\frac{h^{4}}{24} \max _{0 \leqslant t \leqslant 3}|t(t-1)(t-2)(t-3)| \cdot \max _{x_{0} \leqslant x \leqslant x_{3}}\left|f^{(4)}(x)\right|
\end{aligned}
$$

令$h(t)=t(t-1)(t-2)(t-3),t\in [0,3]$.为求解$|h(t)|$的最大值,不妨设$m=(t-1)(t-2),n=t(t-3)$,则易知$m=n+2$.由于$n=(t-\frac 32)^2-\frac 94 \in [-\frac 94,0]$,且$|h(t)|=|mn|=|n(n+2)|=|(n+1)^2-1|$,所以当$n=-1$时$|h(t)|$取最大值$1$.即$\max\limits _{0 \leqslant t \leqslant 3}|t(t-1)(t-2)(t-3)|=1$.因此我们得到

$$
R_3=\max _{x_{0} \leqslant x \leqslant x_{3}}\left|f(x)-L_{3}(x)\right| \leqslant \frac{1}{24} h^{4}\cdot \max _{x_{0} \leqslant x \leqslant x_{3}}\left|f^{(4)}(x)\right|
$$


\end{tcolorbox}


\begin{tcolorbox}[breakable,enhanced,arc=0mm,outer arc=0mm,
		boxrule=0pt,toprule=1pt,leftrule=0pt,bottomrule=1pt, rightrule=0pt,left=0.2cm,right=0.2cm,
		titlerule=0.5em,toptitle=0.1cm,bottomtitle=-0.1cm,top=0.2cm,
		colframe=white!10!biru,colback=white!90!biru,coltitle=white,
            coltext=black,title =2024-03-10, title style={white!10!biru}, before skip=8pt, after skip=8pt,before upper=\hspace{2em},
		fonttitle=\bfseries,fontupper=\normalsize]
  
2. 若 $ f(x)=a_{0}+a_{1} x+\cdots+a_{n-1} x^{n-1}+a_{n} x^{n} $ 有 $ n $ 个不同实根 $ x_{1}, {x}_{2}, {x}_{3}, \cdots $, $ x_{n} $. 证明:

$$
\sum_{j=1}^{n} \frac{x_{j}^{k}}{f^{\prime}\left(x_{j}\right)}=\left\{\begin{array}{ll}
0 & 0 \leq k \leq n-2 \\
a_{n}^{-1} & k=n-1
\end{array}\right.
$$

 \tcblower

 由已知条件知 $ f(x)=a_{n}\left(x-x_{1}\right)\left(x-x_{2}\right) \cdots\left(x-x_{n}\right) $.
记 $ g(x)=x^{k}, \omega_{n}(x)=\prod\limits_{j=1}^{n}\left(x-x_{j}\right) $, 则
$$
f(x)=a_n \prod\limits_{j=1}^{n}\left(x-x_{j}\right)=a_{n} \omega_{n}(x), \quad f^{\prime}\left(x_{j}\right)=a_{n} \omega_{n}^{\prime}\left(x_{j}\right),
$$
$$\begin{aligned}
\text { 其中 } \omega_{n}^{\prime}(x)&=  \left(x-x_{2}\right)\left(x-x_{3}\right) \cdots\left(x-x_{n}\right)+\left(x-x_{1}\right)\left(x-x_{3}\right) \cdots\left(x-x_{n}\right) \\
& +\cdots+\left(x-x_{1}\right)\left(x-x_{2}\right) \cdots\left(x-x_{n-1}\right)\\
&=\left(x_{j}-x_{1}\right)\left(x_{j}-x_{2}\right) \cdots\left(x_{j}-x_{j-1}\right)\left(x_{j}-x_{j+1}\right) \cdots\left(x_{j}-x_{n}\right) 
\end{aligned} $$
由差商的性质可得 $$ g\left[x_{1}, \cdots, x_{n}\right]=\sum\limits_{j=1}^{n}\frac{g(x_j)}{\left(x_{j}-x_{1}\right)\left(x_{j}-x_{2}\right) \cdots\left(x_{j}-x_{j-1}\right)\left(x_{j}-x_{j+1}\right) \cdots\left(x_{j}-x_{k}\right) }=\sum\limits_{j=1}^{k} \frac{g\left(x_{j}\right)}{\omega_{n}^{\prime}\left(x_{j}\right)}  $$
再根据差商与导数之间的关系有$ g\left[x_{1}, \cdots, x_{n}\right]=\frac{f^{(n-1)}(\xi)}{(n-1) !}$,其中$ \xi $ 介于 $ x_{1}, \cdots, x_{n} $ 之间,因此
$$
\begin{aligned}
\sum_{j=1}^{n} \frac{x_{j}^{k}}{f^{\prime}\left(x_{j}\right)} & =\sum_{j=1}^{n} \frac{x_{j}^{k}}{a_{n} \omega_{n}^{\prime}\left(x_{j}\right)}=\frac{1}{a_{n}} \sum_{j=1}^{n} \frac{x_{j}^{k}}{\omega_{n}^{\prime}\left(x_{j}\right)} \\
& =\frac{1}{a_{n}} g\left[x_{1}, x_{2}, \cdots, x_{n}\right] \\
& =\frac{1}{a_{n}} \frac{g^{(n-1)}(\xi)}{(n-1) !},
\end{aligned}
$$
其中 $ \xi $ 介于 $ x_{1}, x_{2}, \cdots, x_{n} $ 之间.
当 $ 0 \leqslant k \leqslant n-2 $ 时, $ g^{(n-1)}(x)=\dfrac{\mathrm{d}^{n-1}}{\mathrm{~d} x^{n-1}} x^{k}=0 $, 故 $ g^{(n-1)}(\xi)=0 $;
当 $ k=n-1 $ 时, $ g^{(n-1)}(x)=\dfrac{\mathrm{d}^{n-1}}{\mathrm{~d} x^{n-1}} x^{n-1}=(n-1) ! $, 故 $ g^{(n-1)}(\xi)=(n-1) ! $, 故有
$$
\sum_{j=1}^{n} \frac{x_{j}^{k}}{f^{\prime}\left(x_{j}\right)}=\frac{1}{a_{n}} \frac{g^{(n-1)}(\xi)}{(n-1) !}=\left\{\begin{array}{ll}
0, & 0 \leqslant k \leqslant n-2 ; \\
a_{n}^{-1}, & k=n-1 .
\end{array}\right.
$$

\end{tcolorbox}

\begin{tcolorbox}[breakable,enhanced,arc=0mm,outer arc=0mm,
		boxrule=0pt,toprule=1pt,leftrule=0pt,bottomrule=1pt, rightrule=0pt,left=0.2cm,right=0.2cm,
		titlerule=0.5em,toptitle=0.1cm,bottomtitle=-0.1cm,top=0.2cm,
		colframe=white!10!biru,colback=white!90!biru,coltitle=white,
            coltext=black,title =2024-03-10, title style={white!10!biru}, before skip=8pt, after skip=8pt,before upper=\hspace{2em},
		fonttitle=\bfseries,fontupper=\normalsize]
  
3. 设 $ x_{0}, x_{1}, \cdots x_{n} $ 为 $ n+1 $ 个互异插值节点, $ {l}_{{0}}({x}), {l}_{{1}}({x}), \cdots, {l}_{{n}}({x}) $ 为 Lagrange 插值基函数, 试证明:

(1) $ \sum\limits_{j=0}^{n} l_{j}(x)=1 $;

(2) $ \sum\limits_{j=0}^{n} x_{j}^{k} l_{j}(x) \equiv x^{k}, k=1,2, \cdots, n $;

(3) $ \sum\limits_{j=0}^{n}\left(x_{j}-x\right)^{k} l_{j}(x)=0, k=1,2, \cdots, n $;

(4) $ \sum\limits_{j=0}^{n} l_{j}(0) x_{j}^{k}=\left\{\begin{array}{ll}1 & k=0 \\ 0 & k=1,2, \cdots, n \\ (-1)^{n} x_{0} x_{1} \cdots x_{n} & k=n+1\end{array}\right. $;
 \tcblower
(1) 令 $ f(x)\equiv1,$ 则 $ y_{j}=f\left(x_{j}\right)=1, j=0,1, \cdots, n $;且函数 $ f(x) $ 的 $ n $ 次Lagrange插值多项式为
$$
L_{n}(x)=\sum_{j=0}^{n}  l_{j}(x)
$$
插值余项为
$$
R_{n}(x)=f(x)-L_{n}(x)=\frac{f^{(n+1)}(\xi)}{(n+1) !}  \omega_{n+1}(x)
$$
因为 $f(x)\equiv 1$,故$f^{(n+1)}(\xi)=0,$于是$R_{n}(x)=0 .$即$f(x)-L_{n}(x)=0$.亦即
$$
\sum_{j=0}^{n}  l_{j}(x)=1
$$

(2) 令 $ f(x)=x^{k}, k=0,1, \cdots, n $. 则函数 $ f(x) $ 的 $ n $ 次插值多项式为
$$
L_{n}(x)=\sum_{j=0}^{n} x_{j}^{k} l_{j}(x)
$$
插值余项为
$$
R_{n}(x)=f(x)-L_{n}(x)=\frac{f^{(n+1)}(\xi)}{(n+1) !}  \omega_{n+1}(x)
$$
因为 $ k \leqslant n $, 则 $ f^{(n+1)}(x)=\dfrac{\mathrm{d}^{n+1}}{\mathrm{~d} x^{n+1}} x^{k}=0 $,
故$f^{(n+1)}(\xi)=0,$于是$R_{n}(x)=0 .$即$f(x)-L_{n}(x)=0$.亦即
$$
\sum_{j=0}^{n} x_{j}^{k} l_{j}(x)=x^{k}, \quad k=0,1, \cdots, n
$$

(3) 对 $ k=1,2, \cdots, n $, 由二项式定理得
$$
\begin{aligned}
\sum_{j=0}^{n}\left(x_{j}-x\right)^{k} l_{j}(x) & =\sum_{j=0}^{n}\left[l_{j}(x) \sum_{i=0}^{k}\left(\begin{array}{l}
k \\
i
\end{array}\right) x_{j}^{i}(-x)^{k-i}\right] \\
& =\sum_{j=0}^{n} \sum_{i=0}^{k}\left[\left(\begin{array}{l}
k \\
i
\end{array}\right) x_{j}^{i}(-x)^{k-i} l_{j}(x)\right] \\
& =\sum_{i=0}^{k} \sum_{j=0}^{n}\left[\left(\begin{array}{l}
k \\
i
\end{array}\right) x_{j}^{i}(-x)^{k-i} l_{j}(x)\right] \\
& =\sum_{i=0}^{k}\left[\left(\begin{array}{l}
k \\
i
\end{array}\right)(-x)^{k-i} \sum_{j=0}^{n} x_{j}^{i} l_{j}(x)\right] \\
& =\sum_{i=0}^{k}\left(\begin{array}{l}
k \\
i
\end{array}\right)(-x)^{k-i} x^{i} \\
& =(x-x)^{k}=0
\end{aligned}
$$


(4) 若函数 $ f(x) $ 在 $ [a, b] $ 上具有 $ n+1 $ 阶导数, 则有
$$
f(x)-L_n(x)=f(x)-\sum_{j=0}^{n} l_{j}(x) f\left(x_{j}\right)=\frac{f^{(n+1)}(\xi)}{(n+1) !} \omega(x)
$$
其中 $ \omega(x)=\left(x-x_{0}\right)\left(x-x_{1}\right) \cdots\left(x-x_{n}\right) $, $\xi \in (a,b)$


当 $ f(x)=1 $ 时,则$1-\sum\limits_{j=0}^{n} l_{j}(x) f\left(x_{j}\right)=0$.
进而有
$$
\sum_{j=0}^{n} l_{j}(0)=1
$$
当 $ f(x)=x^{k}(k=1,2, \cdots, n) $ 时, 有$x^{k}=\sum\limits_{j=0}^{n} l_{j}(x) x_{j}^{k}$.
将 $ x=0 $ 代入得
$$
\sum_{j=0}^{n} l_{i}(0) x_{j}^{k}=0
$$
当 $ f(x)=x^{n+1} $ 时, 有
$$
x^{n+1}=\sum_{j=0}^{n} l_{j}(x) x_{j}^{n+1}+\omega_{n+1}(x)
$$
将 $ x=0 $ 代入得
$$
\sum_{j=0}^{n} l_{j}(0) x_{j}^{n+1}=-\omega_{n+1}(0)=(-1)^{n} x_{0} x_{1} \cdots x_{n}
$$
综上,
$$ \sum\limits_{j=0}^{n} l_{j}(0) x_{j}^{k}=\left\{\begin{array}{ll}1 & k=0 \\ 0 & k=1,2, \cdots, n \\ (-1)^{n} x_{0} x_{1} \cdots x_{n} & k=n+1\end{array}\right. $$

\begin{tcolorbox}[title=补充一个推论(可直接用于上面的证明中)]
   若 $ f(x) $ 是次数不超过 $ n $ 的多项式,则它的 $ n $ 次 Lagrange 插值多项式就是它本身.
\tcblower
证明: 设 $ p_{n}(x) $ 是 $ f(x) $ 的满足插值条件 $ p_{n}\left(x_{i}\right)=f\left(x_{i}\right)(i=0,1,2, \cdots, n) $的 $ n $ 次 Lagrange 插值多项式. 因为 $ f(x) $ 是次数不超过 $ n $ 的多项式, 所以 $ f^{(n+1)}(x) \equiv 0 $.
则 $ p_{n}(x) $ 关于 $ f(x) $ 的余项为
$$
r_{n}(x)=f(x)-p_{n}(x)=\frac{\omega_{n+1}(x)}{(n+1) !} f^{(n+1)}(\xi) \equiv 0,
$$
于是 $ p_{n}(x) \equiv f(x) $. 证毕.
\end{tcolorbox}

\end{tcolorbox}




\begin{tcolorbox}[breakable,enhanced,arc=0mm,outer arc=0mm,
		boxrule=0pt,toprule=1pt,leftrule=0pt,bottomrule=1pt, rightrule=0pt,left=0.2cm,right=0.2cm,
		titlerule=0.5em,toptitle=0.1cm,bottomtitle=-0.1cm,top=0.2cm,
		colframe=white!10!biru,colback=white!90!biru,coltitle=white,
            coltext=black,title =2024-03-10, title style={white!10!biru}, before skip=8pt, after skip=8pt,before upper=\hspace{2em},
		fonttitle=\bfseries,fontupper=\normalsize]
  
  4. 设 $ p_{n}(x) $ 是 $ e^{x} $ 在区间 $ [0,1] $ 上的 Lagrange 型插值多项式,插值节点 $ x_{k}=\frac{k}{n}, k=0,1, \cdots, n $. 证明:
$$
\lim _{n \rightarrow \infty} \max _{0 \leq x \leq 1}\left|p_{n}(x)-e^{x}\right|=0 .
$$

 \tcblower
【证明】 设 $ f(x)=e^x, x \in[0,1],$于是插值余项$R_n(x)=f(x)-p_{n}(x)$,即

$$ f(x)-p_{n}(x)=\frac{f^{(n+1)}(\xi)}{(n+1) !} \prod_{i=0}^{n}\left(x-x_{i}\right) \quad \xi \in(\min \{x, 0\}, \max \{x, 1\}) $$
$$
f^{\prime}(x)=e^x, \quad f^{\prime \prime}(x)=e^x ,\quad f^{\prime \prime \prime}(x)=e^x, \cdots, f^{(n+1)}(x)=e^x
$$

当 $ x \in[0,1] $ 时, $ \left|f^{(n+1)}(x)\right| \leqslant e $ ,
且 $ \left|x-x_{i}\right| \leqslant 1, i=0,1,2, \cdots, n $.
于是当 $ x \in[0,1] $ 时, 有
$$
\left|f(x)-p_{n}(x)\right| \leqslant \frac{e}{(n+1) !}
$$
所以
$$
\max _{0 \leqslant x \leqslant 1}\left|p_n(x)-e^x\right| \leqslant \frac{e}{(n+1) !}
$$
$$
\lim _{n \rightarrow \infty} \max _{0 \leqslant x \leqslant 1}\left|p_n(x)-e^x\right| \leqslant \lim _{n \rightarrow \infty} \frac{e}{(n+1) !}=0 .
$$

\end{tcolorbox}


\begin{tcolorbox}[breakable,enhanced,arc=0mm,outer arc=0mm,
		boxrule=0pt,toprule=1pt,leftrule=0pt,bottomrule=1pt, rightrule=0pt,left=0.2cm,right=0.2cm,
		titlerule=0.5em,toptitle=0.1cm,bottomtitle=-0.1cm,top=0.2cm,
		colframe=white!10!biru,colback=white!90!biru,coltitle=white,
            coltext=black,title =2024-03-10, title style={white!10!biru}, before skip=8pt, after skip=8pt,before upper=\hspace{2em},
		fonttitle=\bfseries,fontupper=\normalsize]
  
5. 设 $ {f}({x}) $ 在 $ [{a}, {b}] $ 上二阶导数连续, 且 $ {f}({a})=0, {f}({b})=0 $,证明:
$$
\max _{a \leq x \leq b}|f(x)| \leq \frac{(b-a)^{2}}{8} \max _{a \leq x \leq b}\left|f^{\prime \prime}(x)\right|
$$
 \tcblower
\textbf{方法一:}
由于 $ f(x) $ 在 $ [a, b] $ 上二阶导数连续, 所以 $ f(x) $ 在 $ [a, b] $ 连续, 根据最值定理知 $f$ 在 $ [a, b] $ 取得最大值和最小值.于是存在点 $ x_0 \in[a, b] $, 使得
$$
|f(x_0)|=\max _{x \in[a, b]}|f(x)| .
$$

  若 $ x_{0}=a $ 或 $ b $, 则结论显然成立. 
 
 若 $ a<x_{0}<b $,分析如下:当 $ f(x_0)>0 $ 时, 根据 $ f(x) \leq|f(x)| \leq f(x_0) $ 可知 $ f(x_0) $ 为 $ f(x) $ 在 $ [a, b] $ 上的最大值; 当 $ f(x_0)<0 $ 时, 根据 $ -f(x) \leq|f(x)| \leq-f(x_0) $ 可知 $ f(x) \geq f(x_0) $, 即 $ f(x_0) $ 为 $ f(x) $ 在 $ [a, b] $ 上的最小值. 总而言之, $ f(x_0) $ 必定为 $ f(x) $ 的最值, 再结合 $ x_0 \in(a, b) $, 就有 $ f^{\prime}(x_0)=0 $.
 
 利用带 Lagrange 余项的 Taylor 公式将 $ f(x) $ 在点 $ x_{0} $ 展开:
$$
\begin{array}{l}
0=f(a)=f\left(x_{0}\right)+f^{\prime}\left(x_{0}\right)\left(a-x_{0}\right)+\frac{1}{2} f^{\prime \prime}(\xi)\left(a-x_{0}\right)^{2}, \xi \in\left(a, x_{0}\right), \\
0=f(b)=f\left(x_{0}\right)+f^{\prime}\left(x_{0}\right)\left(b-x_{0}\right)+\frac{1}{2} f^{\prime \prime}(\eta)\left(b-x_{0}\right)^{2}, \eta \in\left(x_{0}, b\right),
\end{array}
$$

将 $f^{\prime}\left(x_{0}\right)=0 $ 代入上面两式, 并进行如下讨论

若 $ a<x_0 \leq \frac{a+b}{2} $, 可知 $$ |f(x_0)|=\left|-\frac{f^{\prime \prime}(\xi)}{2}(a-x_0)^{2}\right| \leq \frac{(b-a)^{2}}{8} \max _{x \in[a, b]}\left|f^{\prime \prime}(x)\right| $$
若 $ \frac{a+b}{2} \leq x_0<b $, 可知 $$ |f(x_0)|=\left|-\frac{f^{\prime \prime}(\eta)}{2}(b-x_0)^{2}\right| \leq \frac{(b-a)^{2}}{8} \max _{x \in[a, b]}\left|f^{\prime \prime}(x)\right| $$

综上可知 $$ \max _{a \leq x \leq b}|f(x)|=|f(x_0)| \leq \frac{(b-a)^{2}}{8} \max _{a \leq x \leq b}\left|f^{\prime \prime}(x)\right| $$

\textbf{方法二:}
以$ x=a $ 和 $ x=b $ 为插值节点,作函数 $ f(x) $ 的一次插值多项式:
$$
L_{1}(x)=f(a) \frac{x-b}{a-b}+f(b) \frac{x-b}{b-a},
$$

因为 $ f(a)=f(b)=0 $ ,则有 $ L_{1}(x)=0 $ ,且插值多项式$L_{1}(x)$的余项
$$
R_1(x)=f(x)-L_{1}(x)=\frac{f^{\prime \prime}(\xi)}{2}(x-a)(x-b),
$$

其中 $ \xi \in(\min \{x, a\}, \max \{x, b\}) $. 因而,
$$
f(x)=\frac{f^{\prime \prime}(\xi)}{2}(x-a)(x-b), x \in[a, b], \xi \in(a, b),
$$

当 $ x \in[a, b] $ 时,
$$
\begin{aligned}
|f(x)| &\leq \max_{x\in[a,b]} \left|\frac{f^{\prime \prime}(x)}{2}(x-a)(x-b)\right| \\
&\leq \frac{1}{2} \max _{x \in[a, b]}\left|f^{\prime \prime}(x)\right| \cdot \max _{x \in[a, b]}|(x-a)(x-b)| \\
&=\frac{(b-a)^{2}}{8} \max _{x \in[a, b]}\left|f^{\prime \prime}(x)\right|
\end{aligned}
$$

于是,有:
$$
\max _{a \leq x \leq b}|f(x)| \leq \frac{(b-a)^{2}}{8} \max _{a \leq x \leq b}\left|f^{\prime \prime}(x)\right| .
$$
\end{tcolorbox}



\begin{tcolorbox}[breakable,enhanced,arc=0mm,outer arc=0mm,
		boxrule=0pt,toprule=1pt,leftrule=0pt,bottomrule=1pt, rightrule=0pt,left=0.2cm,right=0.2cm,
		titlerule=0.5em,toptitle=0.1cm,bottomtitle=-0.1cm,top=0.2cm,
		colframe=white!10!biru,colback=white!90!biru,coltitle=white,
            coltext=black,title =2024-03-10, title style={white!10!biru}, before skip=8pt, after skip=8pt,before upper=\hspace{2em},
		fonttitle=\bfseries,fontupper=\normalsize]
  
6. 设 $ S(x) $ 是函数 $ f(x) $ 在区间 $ [0,2] $ 上满足第一类条件的三次样条, 并且
$$
S(x)=\left\{\begin{array}{ll}
2 x^{3}-3 x+4, & 0 \leq x<1, \\
(x-1)^{3}+b(x-1)^{2}+c(x-1)+3, & 1 \leq x \leq 2,
\end{array}\right.
$$
求 $ S^{\prime}(0) $ 和 $ S^{\prime}(2) $ 的值.
 \tcblower

取 $ x_{0}=0, x_{1}=1, x_{2}=2 $, 根据三次样条函数的定义, 有 $ S(x) \in C^{2}[0,2] $, 由 $ S(x) $ 及其导数的连续性, 即有: 
$$ S\left(x_{1}-0\right)=S\left(x_{1}+0\right), S^{\prime}\left(x_{1}-0\right)=S^{\prime}\left(x_{1}+0\right), S^{\prime \prime}\left(x_{1}-0\right)=S^{\prime \prime}\left(x_{1}+0\right) $$ 
或者写成如下形式:
$$
\left\{\begin{array}{l}
\lim\limits _{x \rightarrow 1^{+}} S(x)=\lim\limits _{x \rightarrow 1^{-}} S(x) \\
\lim \limits_{x \rightarrow 1^{+}} S^{\prime}(x)=\lim\limits _{x \rightarrow 1^{-}} S^{\prime}(x)\\
\lim \limits_{x \rightarrow 1^{+}} S^{\prime\prime}(x)=\lim\limits _{x \rightarrow 1^{-}} S^{\prime\prime}(x)
\end{array}\right.
$$

而$ \begin{array}{l}S^{\prime}(x)=\left\{\begin{array}{ll}6 x^{2}-3 , &0 \leqslant x<1 \\ 3(x-1)^{2}+2 b(x-1)+c,  &1 \leqslant x \leqslant 2\end{array}\right. \quad S^{\prime \prime}(x)=\left\{\begin{array}{ll}12 x & 0 \leqslant x<1 \\ 6(x-1)+2 b & 1 \leqslant x \leqslant 2\end{array}\right. \end{array} $

由此可得
$$
\left\{\begin{array}{l}
3=3 \\
3=c \\
12=2b
\end{array}\right.
$$

解得$b=6, c=3$
于是有 $ S^{\prime}(0)=-3 $ 和 $ S^{\prime}(2)=18 $.
\end{tcolorbox}


\begin{tcolorbox}[breakable,enhanced,arc=0mm,outer arc=0mm,
		boxrule=0pt,toprule=1pt,leftrule=0pt,bottomrule=1pt, rightrule=0pt,left=0.2cm,right=0.2cm,
		titlerule=0.5em,toptitle=0.1cm,bottomtitle=-0.1cm,top=0.2cm,
		colframe=white!10!biru,colback=white!90!biru,coltitle=white,
            coltext=black,title =2024-03-10, title style={white!10!biru}, before skip=8pt, after skip=8pt,before upper=\hspace{2em},
		fonttitle=\bfseries,fontupper=\normalsize]
  
7. 已知函数表如下:
\begin{tabular}{|c|c|c|c|c|c|}
\hline$ x $ & $-2$ & $-1$ & 0 & 1 & 2 \\
\hline$ f(x) $ & 0 & 1 & 2 & 1 & 0 \\
\hline
\end{tabular}
用二次曲线拟合表中数据.

 \tcblower
\textbf{方法一:}对于给定的一组数据 $ \left(x_{i}, y_{i}\right), i=0,1, \cdots, m $, 求作 $ n $ 次多项式
$$
y=\sum_{k=0}^{n} a_{k} x^{k}
$$
使残差平方和为最小, 于是有法(正则)方程组
$$
\left[\begin{array}{ccccc}
m & \sum x_{i} & \sum x_{i}^{2} & \cdots & \sum x_{i}^{n} \\
\sum x_{i} & \sum x_{i}^{2} & \sum x_{i}^{3} & \cdots & \sum x_{i}^{n+1} \\
\vdots & \vdots & \vdots & & \vdots \\
\sum x_{i}^{n} & \sum x_{i}^{n+1} & \sum x_{i}^{n+2} & \cdots & \sum x_{i}^{2 n}
\end{array}\right]\left[\begin{array}{c}
a_{0} \\
a_{i} \\
\vdots \\
a_{n}
\end{array}\right]=\left[\begin{array}{c}
\sum y_{i} \\
\sum x_{i} y_{i} \\
\vdots \\
\sum x_{i}^{n} y_{i}
\end{array}\right]
$$
其中 $ \sum $ 是 $ \sum\limits_{i=1}^{n} $ 的简写.求出 $ a_{0}, a_{1}, \cdots, a_{n} $, 就得到了拟合多项式的系数.

设二次拟合函数 $ y=a_{0}+a_{1} x+a_{2} x^{2} $.于是根据上面我们有

$$\left[\begin{array}{ccl}5 & \sum\limits_{i=1}^{5} x_{i} & \sum\limits_{i=1}^{5} x_{i}^{2} \\ \sum\limits_{i=1}^{5} x_{i} & \sum\limits_{i=1}^{5} x_{i}^{2} & \sum\limits_{i=1}^{5} x_{i}^{3} \\ \sum\limits_{i=1}^{5} x_{i}^{2} & \sum\limits_{i=1}^{5} x_{i}^{3} & \sum\limits_{i=1}^{5} x_{i}^{4}\end{array}\right]\left[\begin{array}{l}a_{0} \\ a_{1} \\ a_{2}\end{array}\right]=\left[\begin{array}{c}\sum\limits_{i=1}^{5} y_{i} \\ \sum\limits_{i=1}^{5} x_{i} y_{i} \\ \sum\limits_{i=1}^{5} x_{i}^{2} y_{i}\end{array}\right]$$
$$  \text { 即 }\left[\begin{array}{ccc}5 & 0 & 10 \\ 0 & 10 & 0 \\ 10 & 0 & 34\end{array}\right]\left[\begin{array}{l}a_{0} \\ a_{1} \\ a_{2}\end{array}\right]=\left[\begin{array}{l}4 \\ 0 \\ 2\end{array}\right]  $$
计算得$a_{0}=1.6571 \quad a_{1}=0 \quad a_{2}=-0.4286$.
因此拟合多项式为
$$
y=1.6571-0.4286 x^{2}
$$

\textbf{方法二 :}
设二次拟合函数 $ y=a_{0}+a_{1} x+a_{2} x^{2} $.
正则方程组 $ \boldsymbol{A}^{\top} \boldsymbol{A} \boldsymbol{\alpha}=\boldsymbol{A}^{\top} \boldsymbol{Y} $
$$
\left[\begin{array}{rrrrr}
1 & 1 & 1 & 1 & 1 \\
-2 & -1 & 0 & 1 & 2 \\
4 & 1 & 0 & 1 & 4
\end{array}\right]\left[\begin{array}{rrr}
1 & -2 & 4 \\
1 & -1 & 1 \\
1 & 0 & 0 \\
1 & 1 & 1 \\
1 & 2 & 4
\end{array}\right]\left[\begin{array}{l}
a_{0} \\
a_{1} \\
a_{2}
\end{array}\right]=\left[\begin{array}{rrrrr}
1 & 1 & 1 & 1 & 1 \\
-2 & -1 & 0 & 1 & 2 \\
4 & 1 & 0 & 1 & 4
\end{array}\right]\left[\begin{array}{l}
0 \\
1 \\
2 \\
1 \\
0
\end{array}\right]
$$
化简为
$$
\left[\begin{array}{rrr}
5 & 0 & 10 \\
0 & 10 & 0 \\
10 & 0 & 34
\end{array}\right]\left[\begin{array}{l}
a_{0} \\
a_{1} \\
a_{2}
\end{array}\right]=\left[\begin{array}{l}
4 \\
0 \\
2
\end{array}\right]
$$
计算得$a_{0}=1.6571 \quad a_{1}=0 \quad a_{2}=-0.4286$.因此拟合多项式为
$$
y=1.6571-0.4286 x^{2}
$$

\textbf{方法三:}
等价于求解超定方程组
$$
\left[\begin{array}{rrr}
1 & -2 & (-2)^{2} \\
1 & -1 & (-1)^{2} \\
1 & 0 & 0^{2} \\
1 & 1 & 1^{2} \\
1 & 2 & 2^{2}
\end{array}\right]\left[\begin{array}{l}
a_{0} \\
a_{1} \\
a_{2}
\end{array}\right]=\left[\begin{array}{l}
0 \\
1 \\
2 \\
1 \\
0
\end{array}\right]
$$
的最小二乘解, 即正则方程组 $ \boldsymbol{A}^{\mathrm{T}} \boldsymbol{A} \boldsymbol{\alpha}=\boldsymbol{A}^{\mathrm{T}} \boldsymbol{Y} $ 的解.
计算得
$$
\boldsymbol{A}^{\top} \boldsymbol{A}=\left[\begin{array}{rrr}
5 & 0 & 10 \\
0 & 10 & 0 \\
10 & 0 & 34
\end{array}\right], \boldsymbol{A}^{\top} \boldsymbol{Y}=\left[\begin{array}{l}
4 \\
0 \\
2
\end{array}\right]
$$
解得 $ \alpha=\left(\begin{array}{lll}1.6571 & 0 & -0.4286\end{array}\right)^{\mathrm{\top}} $, 拟合多项式
$$
y=1.6571-0.4286 x^{2}
$$
均方误差
$$
\delta=\sum_{i=1}^{5}\left[y\left(x_{i}\right)-y_{i}\right]^{2}=0.13912813
$$
\end{tcolorbox}

\begin{tcolorbox}[breakable,enhanced,arc=0mm,outer arc=0mm,
		boxrule=0pt,toprule=1pt,leftrule=0pt,bottomrule=1pt, rightrule=0pt,left=0.2cm,right=0.2cm,
		titlerule=0.5em,toptitle=0.1cm,bottomtitle=-0.1cm,top=0.2cm,
		colframe=white!10!biru,colback=white!90!biru,coltitle=white,
            coltext=black,title =2024-03-10, title style={white!10!biru}, before skip=8pt, after skip=8pt,before upper=\hspace{2em},
		fonttitle=\bfseries,fontupper=\normalsize]
  
8. 求一个次数不超过 4 次的插值多项式 $ p(x) $, 使它满足:
$$
\begin{array}{l}
p(0)=f(0)=0, p(1)=f(1)=1, p^{\prime}(0)=f^{\prime}(0)=0, \\
p^{\prime}(1)=f^{\prime}(1)=1, p^{\prime \prime}(1)=f^{\prime \prime}(1)=0
\end{array}
$$
并求其余项表达式(设 $ f(x) $ 存在 5 阶导数).
 \tcblower
为了找到一个次数不超过 4 次的插值多项式 $p(x)$,我们可以利用这些插值条件来构建插值多项式.

设 $p(x) = ax^4 + bx^3 + cx^2 + dx + e$,我们需要找到参数 $a, b, c, d, e$ 来满足给定的插值条件.根据插值条件,我们有:
$$
\begin{cases}
p(0) = e = 0 \\
p(1) = a + b + c + d + e = 1 \\
p'(0) = d = 0 \\
p'(1) = 4a + 3b + 2c + d = 1 \\
p''(1) = 12a + 6b + 2c = 0
\end{cases}
$$
解这个方程组,可以得到 $a = 1$,$b = -3$,$c = 3$,$d = 0$,$e = 0$.
因此,插值多项式为 $p(x) = x^4 -3x^3 +3x^2$.

由于$ f(x) $ 在区间$[0,1]$存在 5 阶导数,且注意到$x=0$为二重节点,$x=1$为三重节点.故插值余项为
$$R(x) = f(x) - p(x)=\frac{1}{5 !} f^{(5)}(\xi) x^{2}(x-1)^{3}, \xi \in[0,1]$$


\colorbox{yellow}{方法二}

根据插值条件,我们设插值多项式为 $ p(x)=x^{2}(ax^{2}+bx+c) $. 解得 $ a=1, b=-3, c=3 $.

设$x_0=0,x_1=1$,为求出余项 $ R(x)=f(x)-p(x) $, 根据 $ R\left(x_{i}\right)=0 $ , $R^{\prime}\left(x_{i}\right)=0(i=0,1) $ 和$R^{\prime\prime}\left(x_{1}\right)=0 $, 设
$$
R(x)=K(x)\left(x-x_{0}\right)^2\left(x-x_{1}\right)^{3}
$$
为确定 $ K(x) $, 构造
$$
\varphi(t)=f(t)-p(t)-K(x)\left(x-x_{0}\right)^2\left(x-x_{1}\right)^{3}
$$
显然 $\varphi(x)=0, \varphi\left(x_{i}\right)=0, i=0,1 $, 且 $\varphi^{\prime}\left(x_{1}\right)=0,\varphi^{\prime\prime}\left(x_{1}\right)=0 $, 

反复应用罗尔定理得 $ \varphi^{(5)}(t) $ 在区间 $ [0, 1] $ 上至少有一个零点 $ \xi $, 故有
$$
\varphi^{(5)}(\xi)=f^{(5)}(\xi)-5 ! K(x)=0
$$
于是
$$
K(x)=\frac{1}{5 !} f^{(5)}(\xi)
$$
故余项表达式
$$
R(x)=\frac{1}{5 !} f^{(5)}(\xi)\left(x-x_{0}\right)^2\left(x-x_{1}\right)^{3}=\frac{f^{(5)}(\xi)}{5!}x^{2}(x-1)^{3}
$$


\end{tcolorbox}



\begin{tcolorbox}[breakable,title=定理]
 设 $ f^{(n)}(x) $ 在 $ [a, b] $ 上连续, $ f^{(n+1)}(x) $ 在 $ (a, b) $ 内存在, 节点 $ a \leqslant x_{0}<x_{1}<\cdots< $ $ x_{n} \leqslant b, L_{n}(x) $ 是满足条件 $L_{n}\left(x_{j}\right)=y_{j}, \quad j=0,1, \cdots, n$ 的插值多项式, 则对任何 $ x \in[a, b] $, 插值余项
$$
R_{n}(x)=f(x)-L_{n}(x)=\frac{f^{(n+1)}(\xi)}{(n+1) !} \omega_{n+1}(x),
$$
这里 $ \xi \in(a, b) $ 且依赖于 $ x$, $\omega_{n+1}(x)=\left(x-x_{0}\right)\left(x-x_{1}\right) \cdots\left(x-x_{n}\right)$.

\tcblower
证明: 由给定条件知 $ R_{n}(x) $ 在节点 $ x_{k}(k=0,1, \cdots, n) $ 上为零, 即 $ R_{n}\left(x_{k}\right)=0(k=0 $, $ 1, \cdots, n) $,于是

$$
R_{n}(x)=K(x)\left(x-x_{0}\right)\left(x-x_{1}\right) \cdots\left(x-x_{n}\right)=K(x) \omega_{n+1}(x),
$$
其中 $ K(x) $ 是与 $ x $ 有关的待定函数.
现把 $ x $ 看成 $ [a, b] $ 上的一个固定点, 作函数
$$
\varphi(t)=f(t)-L_{n}(t)-K(x)\left(t-x_{0}\right)\left(t-x_{1}\right) \cdots\left(t-x_{n}\right),
$$
根据 $ f $ 的假设可知 $ \varphi^{(n)}(t) $ 在 $ [a, b] $ 上连续, $ \varphi^{(n+1)}(t) $ 在 $ (a, b) $ 内存在. 根据插值条件及余项定义, 可知 $ \varphi(t) $ 在点 $ x_{0}, x_{1}, \cdots, x_{n} $ 及 $ x $ 处均为零, 故 $ \varphi(t) $ 在 $ [a, b] $ 上有 $ n+2 $ 个零点, 根据罗尔 (Rolle) 定理, $ \varphi^{\prime}(t) $ 在 $ \varphi(t) $ 的两个零点间至少有一个零点, 故 $ \varphi^{\prime}(t) $ 在 $ [a, b] $ 内至少有 $ n+1 $ 个零点. 对 $ \varphi^{\prime}(t) $ 再应用罗尔定理, 可知 $ \varphi^{\prime \prime}(t) $ 在 $ [a, b] $ 内至少有 $ n $ 个零点. 依此类推, $ \varphi^{(n+1)}(t) $在 $ (a, b) $ 内至少有一个零点, 记为 $ \xi \in(a, b) $, 使
$$
\varphi^{(n+1)}(\xi)=f^{(n+1)}(\xi)-(n+1) ! K(x)=0,
$$
于是
$$
K(x)=\frac{f^{(n+1)}(\xi)}{(n+1) !}, \quad \xi \in(a, b), \text { 且依赖于 } x .
$$
将它代入式$R_n(x)$,就得到余项表达式. 证毕.
\end{tcolorbox}



