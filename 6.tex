\section{常微分方程的数值解法练习题}
   \begin{tcolorbox}[breakable,enhanced,arc=0mm,outer arc=0mm,
		boxrule=0pt,toprule=1pt,leftrule=0pt,bottomrule=1pt, rightrule=0pt,left=0.2cm,right=0.2cm,
		titlerule=0.5em,toptitle=0.1cm,bottomtitle=-0.1cm,top=0.2cm,
		colframe=white!10!biru,colback=white!90!biru,coltitle=white,
            coltext=black,title =2024-04, title style={white!10!biru}, before skip=8pt, after skip=8pt,before upper=\hspace{2em},
		fonttitle=\bfseries,fontupper=\normalsize]

 设有常微分方程初值问题如下:
$
\left\{\begin{array}{c}
y^{\prime}=f(x, y) \\
y\left(x_{0}\right)=y_{0}
\end{array}\right.
$

(1) 试推导数值求解公式 $ y_{i+1}=y_{i-1}+2 hf\left(x_{i}, y_{i}\right) $, 并验证公式的阶数

(2) 在初值问题中取 $ f(x, y)=2 x+3 y, x_{0}=0, y_{0}=2, $试用改进欧拉方法求 $ y(0.1) $ 的近似值, 并利用 (1) 中的公式求 $ y(0.2) $ 的近似值

(3) 求数值计算公式
$y_{i+1}=y_{i}+\frac{h}{2}\left[f\left(x_{i}, y_{i}\right)+f\left(x_{i}+\frac{h}{2}, y_{i}+\frac{h}{2} f\left(x_{i}, y_{i}\right)\right)\right]$的阶数
\tcblower
(1) %使用中心差商$\frac{1}{2h}[y(x_{i+1})-y(x_{i-1})]$替代方程 $y^{\prime}(x_i)=f(x_i, y(x_i))$中的导数项
中心差商近似导数:
$$
y'(x_i) \approx \frac{y(x_{i+1}) - y(x_{i-1})}{2h}
$$
将中心差商代入微分方程,得到:
$$
\frac{y(x_{i+1}) - y(x_{i-1})}{2h} = f(x_i, y(x_i))
$$
通过移项和整理,得到数值求解公式:
$$
y(x_{i+1}) = y(x_{i-1}) + 2hf(x_i, y(x_i))
$$
将 $ y(x_i) $ 用 $ y_i $ 表示,得到:
$$
y_{i+1} = y_{i-1} + 2hf(x_i, y_i)
$$
接下来我们验证公式的阶数.将 $ y = y(x) $ 在 $ x_{i+1} $ 和 $ x_{i-1} $ 点的函数值 $ y(x_{i+1}) $ 和 $ y(x_{i-1}) $ 展开为关于 $ x_i $ 点的泰勒级数:
$$
\begin{array}{l}
y(x_{i+1}) = y(x_i) + hy'(x_i) + \frac{h^2}{2} y''(x_i) + \frac{h^3}{6} y'''(\xi_1), \quad x_i < \xi_1 < x_{i+1} \\
y(x_{i-1}) = y(x_i) - hy'(x_i) + \frac{h^2}{2} y''(x_i) - \frac{h^3}{6} y'''(\xi_2), \quad x_{i-1} < \xi_2 < x_i
\end{array}
$$

将这两个式子相减,得:
$$
y(x_{i+1}) - y(x_{i-1}) = 2h y'(x_i) + \frac{h^3}{6} \left( y'''(\xi_1) + y'''(\xi_2) \right)=2h y'(x_i) + \frac{h^3}{3} y'''(\xi),\quad x_{i-1} < \xi < x_{i+1}
$$
因此有
$$
y(x_{i+1}) = y(x_{i-1}) + 2h y'(x_i) + O(h^3)
$$
因此,局部截断误差为:
$$y(x_{i+1})-y_{i+1}=[ y(x_{i-1}) + 2h y'(x_i) + O(h^3)]-[ y_{i-1} + 2hf(x_i, y_i)]=O(h^3)$$
故该公式为二阶方法.

(2)改进欧拉方法的公式为$\left\{\begin{array}{l}
    \widetilde{y}_{i+1}= y_i + h f(x_i, y_i)  \\
    y_{i+1} = y_i + \frac{h}{2} \left[ f(x_i, y_i) + f\left(x_{i + 1}, \widetilde{y}_{i+1} \right) \right]
\end{array}\right.$或写成如下形式:


$$y_{i+1} = y_i + \frac{h}{2} \left[ f(x_i, y_i) + f\left(x_{i + 1}, y_i + h f(x_i, y_i) \right) \right]
$$

给定初值问题:$\left\{
\begin{array}{l}
y' = 2x + 3y \\
y(0) = 2
\end{array}
\right.$我们需要用改进欧拉方法求  $y(0.1)$ 的近似值,取步长 $h = 0.1$.初始条件:$x_0 = 0, \quad y_0 = 2$, $\quad x_1 = x_0 + h = 0 + 0.1 = 0.1$, $f(x_0, y_0) = 2x_0 + 3y_0 = 2(0) + 3(2) = 6$
$$
\begin{array}{c}
      \widetilde{y}_{1} = y_0 + h f(x_0, y_0) = 2 + 0.1 \cdot 6 = 2 + 0.6 = 2.6 \\
     f(x_1, \widetilde{y}_{1}) = 2x_1 + 3\widetilde{y}_{1} = 2(0.1) + 3(2.6) = 0.2 + 7.8 = 8
\end{array}
$$
改进的 $y$ 值:
$$
   y_1 = y_0 + \frac{h}{2} \left[ f(x_0, y_0) + f(x_1, \widetilde{y}_{1}) \right] = 2 + \frac{0.1}{2} \left[ 6 + 8 \right] = 2 + 0.7 = 2.7
$$

因此,改进欧拉方法下 $ y(0.1) $ 的近似值为 2.7.

我们使用公式 $ y_{i+1} = y_{i-1} + 2hf(x_i, y_i) $ 来求 $ y(0.2) $ 的近似值.

%使用欧拉法计算 $ y_1 $:给定 $ x_0 = 0 $,$ y_0 = 2 $,步长 $ h = 0.1 $.

%$$y_1 = y_0 + h f(x_0, y_0)$$

%计算 $ f(x_0, y_0) $:$f(x_0, y_0) = 2x_0 + 3y_0 = 2 \cdot 0 + 3 \cdot 2 = 6$

%因此,$$y_1 = 2 + 0.1 \cdot 6 = 2 + 0.6 = 2.6$$

使用两步中点法计算 $ y_2 $:现在我们有两个初始点 $ y_0 = 2 $ 和 $ y_1 = 2.7 $.使用两步中点法公式来计算 $ y_2 $:
$$
y_2 = y_0 + 2hf(x_1, y_1)
$$
根据$x_1 = 0.1, \quad y_1 = 2.7$计算得
$$
f(x_1, y_1) = 2x_1 + 3y_1 = 2 \cdot 0.1 + 3 \cdot 2.7 = 0.2 +8.1 = 8.3
$$
因此,
$$
y_2 = y_0 + 2 \cdot 0.1 \cdot 8.3 = 2 + 1.66 = 3.66
$$
所以,使用两步中点法公式求得 $ y(0.2) $ 的近似值为 3.66.

(3)
对方程两边求导, 可得
$$
\begin{aligned}
 y^{\prime}(x)=&f(x, y), \quad y^{\prime \prime}(x)=\frac{\partial f(x, y(x))}{\partial x}+y^{\prime}(x) \frac{\partial f(x, y(x))}{\partial y}, \\
y^{\prime \prime \prime}(x)= & \frac{\partial^{2} f(x, y(x))}{\partial x^{2}}+y^{\prime}(x) \frac{\partial^{2} f(x, y(x))}{\partial x \partial y} \\
& +y^{\prime}(x)\left[\frac{\partial^{2} f(x, y(x))}{\partial x \partial y}+y^{\prime}(x) \frac{\partial^{2} f(x, y(x))}{\partial y^{2}}\right]+y^{\prime \prime}(x) \frac{\partial f(x, y(x))}{\partial y},
\end{aligned}
$$

则
$$
\begin{aligned}
R_{i+1}= & y\left(x_{i+1}\right)-\left[y\left(x_{i}\right)+\frac h2 f(x_i,y_i)+\frac h2 f\left(x_{i}+\frac{h}{2}, y\left(x_{i}\right)+\frac{h}{2} f\left(x_{i}, y\left(x_{i}\right)\right)\right)\right] \\
= & y\left(x_{i+1}\right)-y\left(x_{i}\right)-\frac h2 f(x_i,y_i)-\frac h2 f\left(x_{i}+\frac{h}{2}, y\left(x_{i}\right)+\frac{h}{2} y^{\prime}\left(x_{i}\right)\right) \\
= & h y^{\prime}\left(x_{i}\right)+\frac{h^{2}}{2} y^{\prime \prime}\left(x_{i}\right)+\frac{h^{3}}{6} y^{\prime \prime \prime}\left(x_{i}\right)+O\left(h^{4}\right)-\frac h2 f(x_i,y_i) \\
& -\frac h2\left\{f\left(x_{i}, y\left(x_{i}\right)\right)+\frac{h}{2} \frac{\partial f\left(x_{i}, y\left(x_{i}\right)\right)}{\partial x}+\frac{h}{2} y^{\prime}\left(x_{i}\right) \frac{\partial f\left(x_{i}, y\left(x_{i}\right)\right)}{\partial y}\right. \\
& +\frac{1}{2}\left[\frac{h^{2}}{4} \frac{\partial^{2} f\left(x_{i}, y\left(x_{i}\right)\right)}{\partial x^{2}}+\frac{h^{2}}{2} y^{\prime}\left(x_{i}\right) \frac{\partial^{2} f\left(x_{i}, y\left(x_{i}\right)\right)}{\partial x \partial y}\right.  \left.\left.+\left(\frac{h}{2} y^{\prime}\left(x_{i}\right)\right)^{2} \frac{\partial^{2} f\left(x_{i}, y\left(x_{i}\right)\right)}{\partial y^{2}}\right]+O\left(h^{3}\right)\right\} \\
= & \frac h2 y^{\prime}\left(x_{i}\right)+\frac{h^{2}}{2} y^{\prime \prime}\left(x_{i}\right)+\frac{h^{3}}{6} y^{\prime \prime \prime}\left(x_{i}\right)+O\left(h^{4}\right) \\
& -\frac h2\left\{y^{\prime}\left(x_{i}\right)+\frac{h}{2} y^{\prime \prime}\left(x_{i}\right)+\frac{h^{2}}{8}\left[y^{\prime \prime \prime}\left(x_{i}\right)-y^{\prime \prime}\left(x_{i}\right) \frac{\partial f\left(x_{i}, y\left(x_{i}\right)\right)}{\partial y}\right]+O\left(h^{3}\right)\right\} \\
= &\frac 14 h^2y^{\prime \prime}\left(x_{i}\right)+ \frac{5}{48}h^3 y^{\prime \prime \prime}\left(x_{i}\right)+\frac{1}{16}h^3 y^{\prime \prime}\left(x_{i}\right) \frac{\partial f\left(x_{i}, y\left(x_{i}\right)\right)}{\partial y}+O\left(h^{4}\right),
\end{aligned}
$$

所给求解公式是一个 一阶公式.

  \end{tcolorbox}


     \begin{tcolorbox}[breakable,enhanced,arc=0mm,outer arc=0mm,
		boxrule=0pt,toprule=1pt,leftrule=0pt,bottomrule=1pt, rightrule=0pt,left=0.2cm,right=0.2cm,
		titlerule=0.5em,toptitle=0.1cm,bottomtitle=-0.1cm,top=0.2cm,
		colframe=white!10!biru,colback=white!90!biru,coltitle=white,
            coltext=black,title =2024-04, title style={white!10!biru}, before skip=8pt, after skip=8pt,before upper=\hspace{2em},
		fonttitle=\bfseries,fontupper=\normalsize]
用梯形方法解初值问题 $ \left\{\begin{array}{c}y^{\prime}+y=0 \\ y(0)=1\end{array}\right. $ 证明其近似解为 $ y_{n}=\left(\frac{2-h}{2+h}\right)^{n} $, 并证明: 当 $ h \rightarrow 0 $ 时, 它收敛于原初始问题的精确解 $ y=e^{-x} $
\tcblower
梯形方法是一个数值解常微分方程的方法,其迭代格式为:
$$ y_{n+1} = y_n + \frac{h}{2}(f(x_n, y_n) + f(x_{n+1}, y_{n+1})) $$
其中,$ h $ 是步长,$ f(x, y) $ 是微分方程 $ y' = f(x, y) $ 中的右端函数.

对于给定的初值问题 $ \left\{\begin{array}{l}y^{\prime}+y=0 \\ y(0)=1\end{array}\right. $,我们有 $ f(x, y) = -y $.将其代入梯形方法的迭代格式中,得到:
$$ y_{n+1} = y_n + \frac{h}{2}(-y_n - y_{n+1}) $$
整理得到:$ y_{n+1} = \frac{2-h}{2+h}y_n $. 于是
$$
y_{n+1}=\left(\frac{2-h}{2+h}\right) y_{n}=\left(\frac{2-h}{2+h}\right)^{2} y_{n-1}=\cdots=\left(\frac{2-h}{2+h}\right)^{n+1} y_{0}
$$
因此,我们证明了 $ y_{n}=\left(\frac{2-h}{2+h}\right)^{n} $ 是给定初值问题的近似解.

%另一方面,对$\forall x>0$, 以 $h$ 为步长经过 $n$ 步运算可求得$y(x_n)$的近似值 $y_n$,所以$x=nh,n=\frac xh$.
因为 $ y_{0}=1 $, 故
$$
y_{n}=\left(\frac{2-h}{2+h}\right)^{n}
$$

对于给定的步长 $ h $,经过 $ n $ 步运算后,我们可以得到 $ y(x) $ 的近似值 $ y_n$.在每一步中,我们都会在 $ x $ 的位置上进行计算,因此总共进行 $ n $ 步运算后,我们得到的 $ x $ 的取值为 $ x = nh $.即 $ n=\dfrac{x}{h} $, 代入上式有:
$$
\begin{aligned}
y_{n}&=\left(\frac{2-h}{2+h}\right)^{\frac x  h} \\
\lim _{h \rightarrow 0} y_{n}&=\lim _{h \rightarrow 0}\left(\frac{2-h}{2+h}\right)^{\frac{x}{h}}=\lim _{h \rightarrow 0}\left(1-\frac{2 h}{2+h}\right)^{\frac{x}{h}} \\
&=\lim _{h \rightarrow 0}\left[\left(1-\frac{2 h}{2+h}\right)^{\frac{2+h}{2 h}}\right]^{\frac{2 h}{2+h}\cdot \frac{x}{h}}=\mathrm{e}^{-x}
\end{aligned}
$$
因此,当 $ h \rightarrow 0 $ 时,$ y_{n}=\left(\frac{2-h}{2+h}\right)^{n} $ 收敛于原初值问题的精确解 $ y=e^{-x} $.

  \end{tcolorbox}


     \begin{tcolorbox}[breakable,enhanced,arc=0mm,outer arc=0mm,
		boxrule=0pt,toprule=1pt,leftrule=0pt,bottomrule=1pt, rightrule=0pt,left=0.2cm,right=0.2cm,
		titlerule=0.5em,toptitle=0.1cm,bottomtitle=-0.1cm,top=0.2cm,
		colframe=white!10!biru,colback=white!90!biru,coltitle=white,
            coltext=black,title =2024-04, title style={white!10!biru}, before skip=8pt, after skip=8pt,before upper=\hspace{2em},
		fonttitle=\bfseries,fontupper=\normalsize]
 应用 Taylor 定理构建求解常微分方程初值问题 $ \left\{\begin{array}{l}y^{\prime}=-y^{2} \\ y(0)=1\end{array}\right. $ 的 2 阶近似求解方法
\tcblower

对于给定的初值问题 $ y'(x) = -y(x)^2 $,我们可以对 $ y(x) $ 进行 Taylor 展开:
$$
y(x_{i+1}) = y(x_i) + hy'(x_i) + \frac{h^2}{2}y''(x_i) + O(h^3)
$$

其中,$ h $ 是步长,$ y'(x_i) $ 和 $ y''(x_i) $ 是 $ y(x) $ 在 $ x_i $ 处的导数和二阶导数.

 已知 $ y'(x) = -y^2 $,所以有$y'(x_i) = -y^2(x_i)$.对 $ y'(x) $ 再求导得到$ y''(x) $:
$$
y''(x) = \frac{d}{dx}(-y^2) = -2y \cdot y'(x)
$$
将 $ y'(x) = -y^2 $ 代入上式得到:$y''(x) = -2y \cdot (-y^2) = 2y^3$.

利用 Taylor 展开式在$x_i$处展开:
$$
y(x_{i+1}) = y(x_i) + hy'(x_i) + \frac{h^2}{2}y''(x_i) + O(h^3)
$$

代入计算得到的 $ y'(x_i) $ 和 $ y''(x_i) $:
$$\begin{aligned}
    y(x_{i+1}) &= y(x_i) + h(-y^2(x_i)) + \frac{h^2}{2}(2y^3(x_i)) + O(h^3)\\&= y(x_i) - hy^2(x_i) + h^2y^3(x_i) + O(h^3)
\end{aligned}
$$

因此,二阶近似求解方法为:
$$
y_{i+1} = y_i - hy_i^2 + h^2y_i^3
$$

  \end{tcolorbox}


   \begin{tcolorbox}[breakable,enhanced,arc=0mm,outer arc=0mm,
		boxrule=0pt,toprule=1pt,leftrule=0pt,bottomrule=1pt, rightrule=0pt,left=0.2cm,right=0.2cm,
		titlerule=0.5em,toptitle=0.1cm,bottomtitle=-0.1cm,top=0.2cm,
		colframe=white!10!biru,colback=white!90!biru,coltitle=white,
            coltext=black,title =2024-04, title style={white!10!biru}, before skip=8pt, after skip=8pt,before upper=\hspace{2em},
		fonttitle=\bfseries,fontupper=\normalsize]
导出用 Euler 法求解 $ \left\{\begin{array}{l}y^{\prime}=\lambda y \\ y(0)=1\end{array} \quad(\lambda \neq 0)\right. $ 的公式, 并证明它收敛于初值问题的精确解.
\tcblower

令 $ x_i = i h $,根据欧拉法$ y_{i+1} = y_i + h f(x_i,y_i)$, 则 $ y_{i+1} = y_i + h y'(x_i) $. 对于 $ y'(x) = \lambda y $,有:
$$
y_{i+1} = y_i + h \lambda y_i = y_i (1 + h \lambda)
$$
初值条件为 $ y(0) = 1 $.因此,欧拉法的递推公式可以写成:$y_{i+1} = y_i (1 + h \lambda)$.

我们从初始值 $ y_0 = 1 $ 开始,逐步计算 $ y_1, y_2, \ldots $,可以得到一般公式:
$$
\begin{array}{l}
       y_1 = y_0 (1 + h \lambda) = 1 \cdot (1 + h \lambda)\\
      y_2 = y_1 (1 + h \lambda) = (1 + h \lambda)^2\\
      \cdots\cdots\\
      y_n = (1 + h \lambda)^n
\end{array}
$$

而易知初值问题的精确解为:$y(x) = e^{\lambda x}$, 
%在 $ x = nh $ 处的精确解为:
%$$
%y(nh) = e^{\lambda nh} = \left(e^{\lambda h}\right)^n
%$$
为了证明欧拉法求解的数值解 $ y_n = (1 + h \lambda)^n $ 收敛于精确解 $y(x) = e^{\lambda x}$,我们需要分析 $ (1 + h \lambda)^n $ 和 $ \left(e^{\lambda x}\right)$ 之间的关系.由于$ x = nh $,
%对于给定的步长 $ h $,经过 $ n $ 步运算后,我们可以得到 $ y(x) $ 的近似值 $ y_n$.在每一步中,我们都会在 $ x $ 的位置上进行计算,因此总共进行 $ n $ 步运算后,我们得到的 $ x $ 的取值为 $ x = nh $.即 $ n=\dfrac{x}{h} $, 代入上式有:
利用指数函数的定义,我们知道当 $ h \to 0 $ 时,有:
$$
\lim_{h\to 0}\left(1 + \lambda h \right)^n =\lim_{h\to 0}\left(1 + \lambda h \right)^{\frac{1}{\lambda h}\cdot \lambda h\cdot \frac{x}{h}}= \lim_{h\to 0}\left(1 + \lambda h \right)^{\frac{1}{\lambda h}\cdot \lambda x}= e^{\lambda x} 
$$

因此,当步长 $ h \to 0 $ (即 $ n \to \infty $)时,欧拉法的数值解 $ y_n = (1 + h \lambda)^n $ 收敛于精确解 $ y(x) = e^{\lambda x} $.


  \end{tcolorbox}

%1. Find the exact solution of the initial value problem $u^{{\prime}} = {-} u^{2},u(0) = 1,$ and compare it to the approximate solutions obtained by successive approximations according to Corollary 10.6. Compute the third iterate $u_{2}$ and compare the exact error $u {-} u_{2}$ to the a posteriori error estimate from Corollary 10.6.

%2. Show that the midpoint methpd for the solution of initial value problem $y_{n + 1} = y_{n {-} 1} + 2hf\left( x_{n},y_{n} \right)$ is convergent of order 2

%3. Using the Taylor expansion to construct the approximation method for the initial value problem $\left\{\begin{array}{l} y^{{\prime}} = {-} y^{2} \\ y(0) = 1 \end{array} \right.$ such that the method is convergent of order 2

%4. Using the trapezoidal (梯形) method to solve the initial value problem $\left\{\begin{matrix} y^{{\prime}} + y = 0 \\ y(0) = 1 \end{matrix} \right.$ ,show that the aproximation solution is $y_{n} = {\left( \frac{2 {-} h}{2 + h} \right)}^{n}$ and $\left\{y_{n} \right\}$ is convergent to the accurate solution $y = e^{{-}x}$ as $h {\rightarrow} 0$
